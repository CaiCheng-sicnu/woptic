%%% shared-files/header.tex
%%%
%%%    Header for wien2wannier and woptic UGs.
%%%

\documentclass[
   a4paper,
%   oneside,
   twoside,
%   draft
]{wien}
\usepackage[utf8]{inputenc}
\usepackage[osf,sc,slantedGreek]{mathpazo}
\linespread{1.05}
\usepackage[scaled=1.05]{zlmtt}
\usepackage[T1]{fontenc}
\usepackage[protrusion=true,
            expansion=true,
            auto=true,
%            kerning=true,
            spacing=true,
            tracking=false]
           {microtype}

\usepackage{amsmath, amssymb, graphicx, alphabeta, braket, titlepic,
  xspace, minitoc, longtable, siunitx, fancyvrb, enumitem, tabu,
  upgreek, placeins, maybemath}

\usepackage[breaklinks,colorlinks,
            linkcolor=black,
            citecolor=black,
            urlcolor=black,
            ]{hyperref}

\usepackage[textsize=small,colorinlistoftodos]{todonotes}
% TODOs
\newcommand*\tdContent[2][]{\todo[color=orange!30,#1]{#2}}
\newcommand*\tdRef    [2][]{\todo[color=green!30,#1]{#2}}

% \include{texdefine}
\addtolength{\textwidth}{2.5cm}
\addtolength{\oddsidemargin}{-1cm}
\addtolength{\evensidemargin}{-1.6cm}

\newcommand*\name  [1]{\textsc{#1}}
\newcommand*\file  [1]{\texttt{#1}}
\newcommand*\fileit[1]{\texttt{\textit{#1}}}
\newcommand*\code  [1]{\texttt{#1}}
\newcommand*\codeit[1]{\texttt{\textit{#1}}}
% for defaults and the like
\newcommand*\rem   [1]{\textit{\liningnums{#1}}}
\newcommand*\usgrem[1]{\textit{\textrm{#1}}}

\newcommand{\liningnums}[1]{%
  {\fontfamily{pplx}\selectfont #1}}

% lining numbers should go with uppercase
\let\tmp\MakeUppercase
\renewcommand\MakeUppercase[1]{\fontfamily{pplx}\selectfont \tmp{#1}}

\newcommand\MyTOC{
  {
    \linespread{1.1}
    \tableofcontents
  }
}

\newcommand*{\abbrev}[1]{\textsc{#1}}
\newcommand*{\Abbrev}[1]{
  \expandafter\newcommand\csname#1\endcsname{\abbrev{#1}\xspace}
}

\Abbrev{cgs}
\Abbrev{dos}
\Abbrev{soc}
\Abbrev{dft}
\Abbrev{Dft}
\Abbrev{mlwf}
\Abbrev{Mlwf}
\Abbrev{lapw}
\Abbrev{Lapw}
\Abbrev{dmft}
\Abbrev{Dmft}
\Abbrev{uc}
\Abbrev{bz}
\Abbrev{wf}
\newcommand*\wfs{\wf{}s}

\newcommand*\listnote{%
  This file is read using Fortran list-oriented reads, i.e., items are
  separated by white space.%
}

\newcommand\ProgName[2][]{
  \def\myarg{#1}
  \ifx\empty\myarg
     \ProgNameII{#2}{#2}
  \else
     \ProgNameII{#1}{#2}
  \fi
}
\newcommand\ProgNameII[2]{
  \expandafter\newcommand\csname#2\endcsname{%
    \hyperref[sec:#2]{\file{#1}}%
    \xspace%
  }
  \expandafter\newcommand\csname#2Hd\endcsname{%
    \texttt{#1}%
    \xspace%
  }
}

\newcommand\ProgSec[2][]{
  \def\myarg{#1}
  \ifx\empty\myarg
     \def\separator{}
  \else
     \def\separator{---}
  \fi
  \section[\csname#2Hd\endcsname{} \separator{} #1]
  {\csname#2Hd\endcsname{} \separator{} #1}
  \label{sec:#2}
}

\newcommand*\siunits {\abbrev{si}\xspace}

\newcommand*\case    {\texttt{\itshape case}\xspace}
\newcommand*\deffile {\texttt{\itshape def}\xspace}

\newcommand*\wtow   {\href{http://wien2wannier.github.io/}{\textsc{wien2wannier}}\xspace}
\newcommand*\Wtow   {\href{http://wien2wannier.github.io/}{\textsc{Wien2Wannier}}\xspace}
\newcommand*\woptic {\href{http://woptic.github.io/}{\textsc{woptic}}\xspace}
\newcommand*\Woptic {\href{http://woptic.github.io/}{\textsc{Woptic}}\xspace}
\newcommand*\wien   {\href{http://www.wien2k.at}{\textsc{Wien}2k}\xspace}
\newcommand*\Wien   {\href{http://www.wien2k.at}{\textsc{Wien}2k}\xspace}
\newcommand*\wannier{\href{http://www.wannier.org}{\textsc{wannier90}}\xspace}
\newcommand*\Wannier{\href{http://www.wannier.org}{\textsc{Wannier90}}\xspace}
\newcommand*\xcrys  {\href{http://www.xcrysden.org}{XCrySDen}\xspace}
\newcommand*\vesta  {\href{http://jp-minerals.org/vesta/en/}{\textsc{vesta}}\xspace}
\newcommand*\Vesta  {\href{http://jp-minerals.org/vesta/en/}{\textsc{Vesta}}\xspace}
\newcommand*\gnuplot{\href{http://www.gnuplot.info}{gnuplot}\xspace}

\DeclareMathOperator{\tr}{tr}
\DeclareMathOperator{\sgn}{sgn}
\renewcommand*\Re{\mathop{\operatorname{Re}}}

\newcommand*\dd      {\text{d}}
\newcommand*\ii      {\text{i}}
\newcommand*\ee      {\text{e}}

\newcommand{\mb}{\boldsymbol}

%%%% Including template files %%%%
%%%% User code must set \TemplateDir
\newcommand*\FancyVerbStopString{} % empty line ends template inclusion

%%% begin deep magic %%%
\expandafter\def\expandafter\setCatCode\expandafter{
  \catcode35 = 13 % # = active
}
\expandafter\def\expandafter\resetCatCode\expandafter{
  \catcode35 =  6 % # = argument
}
\setCatCode
\def#{\color{gray}\itshape\char35}
\resetCatCode
\newcommand{\Template}[2][]{%
  \\[\baselineskip]
  \begin{minipage}{\linewidth}
    \VerbatimInput[%
    codes={ \setCatCode },
    label={%
      [\raisebox{-.5\height}{\case.#2}]%
      \raisebox{-.5\height}{end of \case.#2}%
    },
    frame=lines, framesep=3mm,
    numbers=left,
    #1
    ]{\TemplateDir/case.#2}
  \end{minipage}
  \\[\baselineskip]
}
%%% end deep magic %%%

\newenvironment{usage}{%
  \list{}{
    \leftmargin    .5\parindent
    \listparindent 0em
    \itemindent    \listparindent
    \linespread{1.1}
  }
  \item\relax\ttfamily%
}{\endlist}

\newlist{options}{description}{1}
\setlist[options]{
  font=\ttfamily, itemsep=0em,
  labelindent=.5\parindent, leftmargin=!, labelsep=1em
}

\newlist{linesI}{enumerate}{1}
\setlist[linesI]{
  font=\bfseries,
  label={line \arabic*}
}

\newenvironment{lines}{%
  \begin{linesI}%
    \newcommand*\T[1]{\texttt{\textit{##1}}}
    \newcommand*\U[1]{\si{##1}}
    \newenvironment{flin}[3][V]{%
      \newcolumntype V {>{\ttfamily}l}
      \newcolumntype T {>{\ttfamily\itshape}l}
      \newcolumntype U {s}
      \def\argIII{##3}
      \ifx\empty\argIII
      \else
        \def\argIII{ --- ##3}
      \fi
      \item \code{##2}\argIII
      \vskip-.5\baselineskip    % longtabu produces an inordinately
                                % large separation
      \begin{longtabu} to \linewidth { V ##1 X}
    }{%
      \end{longtabu}%
    }%
    % \newcommand{\flinbrk}{% — does not work
    %   \endtabu\endgroup
    %   \begingroup\tabu to \linewidth { *2{ >{\ttfamily}l } X}%
    % }%
  }{%
  \end{linesI}%
}

\setcounter{minitocdepth}{1}
\setcounter{secnumdepth}{1}
\setcounter{tocdepth}{1}
\dominitoc


%%% Local Variables: 
%%% mode: latex
%%% End: 
